\chapter{Mindset}

"Technology involves organization, procedures, symbols, new words, equations, and, most of all, a mindset." -- Ursula Franklin \todo{make this a proper quote}

\section{Some general things to keep in mind first}

\subsection{Respectfully take the newcomer’s lack of knowledge into account}
Think about how you would want to be treated when approaching a new body of knowledge. Maybe programming is easy for you, but you’d like to learn to bake. If the first person you talked to about baking laughed at you for not knowing the difference between a pâte à choux and pâte brisée, you probably wouldn’t be very excited to ask that person to help you further. Keep in mind that for people who are new to free software, even those who are experienced programmers, it often feels like they’re encountering a whole new language. Lead with patience and skip the derision.

Example:
New person: I think I’ve tried everything and it’s just not working!
Crummy response: Ha, ha, ha! It’s soooo easy! I can’t believe you’re not getting this!
Good response: Hmm, I remember this being tricky the first time too. Keep trying, I know you’ll get it! \todo{format this dialog properly}

\subsection{Anticipate the specific words a newcomer might not know}
Be mindful of what newcomers might not know, such as terms like “upstream” and “downstream”, technical and organizational structure of different projects, and any abbreviations.

\subsection{Educate newcomers about online resources and the culture of questioning in our community}
One of the great things about free software is all of the amazing community-provided resources for when you get stuck. Most new contributors aren’t used to going to a chatroom or a forum for technical support. Proprietary software doesn’t encourage asking around for help, so people are used to calling tech support. It’s very empowering to learn that 1) lots of people have had the exact same question as you and 2) the answer has already been provided. No waiting and you found the solution yourself — woo! 

\subsection{Take advantage of the newcomers’ questions}
For your project, document as many of your frequent questions as you can and be grateful for people who are asking new questions. Their outside perspective helps you build better documentation and ultimately a better project. 

Fundamentally, the key to successful community management and outreach is intentional, structured empathetic thinking.

\section{Outreach Can Help Increase Diversity}
Outreach Can Help Increase Diversity
In 2006, when GNOME began an outreach effort specific to finding prospective women contributors, they received 100 applications and many more inquiries. The experience was unusual but thrilling; it came on the heels of a Summer of Code where, of the 181 people who applied, none appeared to be women\footnote{http://thegnomejournal.wordpress.com/2006/08/15/the-womens-summer-outreach-program}. Similarly, when Etsy offered grants for women to participate in Hacker School and raised awareness among women about the opportunity, Hacker School received 661 applications from women, while they only received 7 for the previous batch. 23 women were accepted, while only 1 was accepted for the previous batch\footnote{http://www.etsy.com/blog/news/2012/update-on-the-hacker-grants-program}. These examples show that there are people who want to be a part of our community, but need a special welcome to walk through the door.

This chapter discusses the possible reasons for why women are staying out of free software and how you can engage more women in your community. You can use these techniques to get more people involved from other underrepresented groups and make your community more welcoming for all newcomers. We need to make sure free software is truly open and accessible to everyone.

A 2002 survey of free software communities found that approximately 1.1\% of contributors to our communities are women. Many outreach efforts have taken place since, and the more recent observations place the number of women in the free software community at about 3\%\footnote{http://geekfeminism.wikia.com/wiki/FLOSS}, with more women participating in the communities that made outreach efforts. For example, due to an active women’s outreach program, GNOME’s yearly conference changed from 4\% women in 2009 to 17\% in 2012. A survey of new contributor found 50\% of GNOME’s newcomers who joined and stayed involved in the project were women, by contrast with 6\% as the average for all other participating free software projects\footnote{https://www.gnome.org/news/2013/01/25-women-in-10-free-software-organizations-for-gnomes-outreach-program-for-women/}.

Women do not have high representation in computer science in general. In US, women represent 25\% of all software developers and 18\% of students currently graduating with computer science degrees\footnote{http://www.nsf.gov/statistics/wmpd/2013}. However, even compared to these numbers, 3\% of women in free software shows a drastic underrepresentation.

Possible reasons fewer women get involved in free software on their own are
\begin{itemize}
 \item women are less likely to be encouraged to participate in free software by people in their surroundings
 \item women often prefer to work collaboratively, while a traditional path for getting involved in free software requires solitary exploration
 \item women are not sure if they would be welcome in the free software community, as it is intimidating to join a group in which you are not like the other people
 \item women are not sure if they would be safe in the free software community, as stories of incidents that happen in some communities feed negative stereotypes about free software as a whole being unsafe for women
 \item free software and the opportunities in it are advertised in heavily male-dominated channels, perpetuating the gender imbalance
\end{itemize}

To increase the number of women in your community, be mindful of where and how you advertise your events and opportunities. Be sure to advertise on diversity-neutral channels, such as on computer science department mailing lists (as compared to computer clubs or free software clubs) and on diversity-seeking channels, such as women in computer science groups on campus and local or global women in technology groups. A listing of women in computer science, engineering, and technology groups is maintained on the Geek Feminism wiki, and you are welcome to add to it\footnote{\url{http://geekfeminism.wikia.com/wiki/List_of_women_groups_in_technology}}.

When you advertise an event or opportunity, talk about it in a collaborative tone, e.g.  “Learn while working with a mentor,” rather than in a competitive tone, e.g. “Prove you are a rockstar”. Make sure you don’t have anything in your advertising copy that suggests that you expect most or all attendees to be heterosexual men, such as using male pronouns or promising the presence of beautiful women to serve attendees beer. This example might sound absurd, but similar things have been advertised by various conferences and hackathons in the past. Take a second look at your advertisement to check for more subtle stereotypical expressions, or ask a woman in your community if she could look over it.

To get even more women interested in free software, you can organize an introductory workshop about it through a women in computer science group on your local university campus. The event can be open to everyone, but putting it together with the help of a women’s group will ensure many women will attend. OpenHatch has excellent curricular and other resources for such a workshop which you can use freely\footnote{http://campus.openhatch.org}.

The GNOME Foundation’s Outreach Program for Women has expanded to include multiple free software organizations, with 12 organizations participating in the 2013 June - September round. It has proven to be a successful way to bring in talented contributors and provide them with a focused opportunity to build up their free software expertise. You can get your organization to join the program\footnote{https://live.gnome.org/OutreachProgramForWomen}.

Adopt an anti-harassment policy for your conferences and events, such as the one created by the Ada Initiative\footnote{https://adainitiative.org/what-we-do/conference-policies}, and create a code of conduct for your community. Hold everyone to high standards in appropriate and respectful communication, and say something privately or publicly if you see or hear someone saying things that are out-of-line.

Encourage women in your community to blog, attend conferences, and give talks. It’s important to give visibility to women who are doing good work in your community. Women are often more critical of their level of expertise on a subject and more shy in proposing talks for conferences\footnote{http://geekfeminism.org/2012/05/21/how-i-got-50-women-speakers-at-my-tech-conference}. It’s helpful to encourage women to speak at conferences to both give them more confidence about their level of achievement and allow them to serve as role models for others. Even presentations by relative beginners are useful because they are often the most accessible talks for other beginners and they can provide a fresh perspective.

\section{How to show you care about newcomers}
Projects that want new people need to explicitly state they do want new people and that they will treat them with respect when they arrive. Your project is competing for the time that people might otherwise spend relaxing or hanging out with friends. Most people will not willingly sign up to be belittled or harangued in their leisure time.  

When hosting a newcomer-focused event, you need to explicitly state that your event is for newcomers. If your general-audience event is beginner-friendly be sure to mention that in all your announcements. You should also consider where the best place for beginners to ask questions online is. Large projects should have a separate IRC channel for beginner level questions or an “all questions welcome” mailing list for interns or other new mentees and designated friendly mentors to go to with 101 questions. 

Some examples include:\todo{make this a proper list}
- the GNOME Love mailing list, IRC channel, and mentors list\footnote{\url{https://live.gnome.org/GnomeLove\#Getting_Help_and_Advice}}
- the Python Core Mentorship mailing list\footnote{\url{http://pythonmentors.com}}

Codes of conduct and diversity statements help you broadcast to newcomers that your group or project is friendly. These statements are an expression of your community’s feelings about newcomers, and you can tailor them to your community as much or as little as you like.

Making sure that your mailing list and IRC conversations conform to a code of conduct will prevent you from alienating lurkers before they even make contact. Mailing lists and IRC channels do not have the benefit of facial expressions or nonverbal cues; inside jokes or familiar modes of interaction among longtime community members may hurt the feelings of newcomers unaware of the context. When you do have missteps or incidents, make sure your response to poor conduct is swift and in the same medium. People who observed the poor conduct should also know about the resolution of the issue.

These are some good examples
\begin{itemize}
  \item Free Geek Chicago is an in-person collaborative space, and their Code of Conduct\footnote{http://codex.freegeekchicago.org/wiki/FreeGeekInfo/Policies/CodeOfConduct} is short, just five paragraphs long. It takes a matter-of-fact approach, focusing on helping members see which behaviors are (and aren’t) helpful for the organization’s goals.
  \item The Ubuntu Code of Conduct\footnote{http://www.ubuntu.com/project/about-ubuntu/conduct} does a great job of describing the type of community they are cultivating and then includes specific suggestions for ways that individuals can help build a great community.
  \item PyCon US's Code of Conduct\footnote{https://us.pycon.org/2012/codeofconduct/} has lots of great ideas about how to run an in-person event, including language about best behavior for an exhibitor floor, concrete steps on how attendees can report unwelcome behavior, and how to handle those reports as a volunteer.
  \item KDE’s Code of Conduct\footnote{http://www.kde.org/code-of-conduct/} goes into depth on what collaboration, respect, being considerate, and being pragmatic, and being supportive mean to its global, decentralized community and lays out many great practices for achieving that.
\end{itemize}

\section{Expect slow growth, not overnight results}
Your outreach is going to take some time and effort — some forms more so than others. Please be aware that not all of it will bear results for you. You’ll have to try and try again and maybe change a few things or try something completely different. But that’s okay. It’s the same for everyone doing outreach. However that doesn’t mean you shouldn’t do it. On the contrary, everything you do will have an effect, even if it is not immediately visible to you. Some of these strategies take years to pay off and might pay off in completely different places and ways than you expected. For example, a student who participated in the Fedora design bounty program eventually became the graphic design lead at MediaGoblin, a totally different free software project. Make sure everyone involved understands this and is not disappointed because of unrealistic expectations.
