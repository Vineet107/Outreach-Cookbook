\chapter{Introduction}

"A person who never made a mistake never tried anything new." -- Albert Einstein \todo{make this a proper quote}

This book is for free software project maintainers and people who run meetups, events or groups that are specifically interested in bringing new people to free software. After reading this book, we hope you’ll be able to help new people, and especially a more diverse group of people, feel welcome in your project or group. Successful outreach is the best way to grow your project and the free software community. We hope you agree that this is an important goal! 

To do successful outreach you will need a few things. The most important resource is probably patience. You will find yourself answering very simple questions over and over again. You will also need a willingness to ask others to join you because you won’t be able to run successful outreach events and campaigns by yourself. And last but not least, you obviously need a desire to get others to join your community.

Running outreach events can be very rewarding! But please don’t think they are a magic wand for getting a million new expert contributors. Bringing new community members up to speed is a lot of work. Many people may wander off after your workshop or hackathon, never to be heard from again. Think of these events like a community service. You will get the warm fuzzy glow of having introduced new people to the idea of free software. While not all participants will stay with your project, some of them might join other projects in the wider free software community. If other projects and meetup groups also run outreach events, everyone benefits from increased energy and growth in the free software community. What you will gain by running outreach events is the joy of participating in a cultural shift. When everyone in your project is thinking about outreach, your project becomes more accessible and friendly.

We hope you will enjoy doing community outreach as much as we do and wish you good luck!

\bookauthors \todo{format properly, right align}